\documentclass[11pt]{article}
\usepackage[margin=1in]{geometry}                % See geometry.pdf to learn the layout options. There are lots.
\geometry{letterpaper}                   % ... or a4paper or a5paper or ... 
%\geometry{landscape}                % Activate for for rotated page geometry
%\usepackage[parfill]{parskip}    % Activate to begin paragraphs with an empty line rather than an indent
\usepackage{graphicx}
\usepackage{amssymb}
\usepackage{epstopdf}
\DeclareGraphicsRule{.tif}{png}{.png}{`convert #1 `dirname #1`/`basename #1 .tif`.png}
\usepackage{authblk}

%For final version of Special issue article: For each paper in the special issue, up to 20 single column double-spaced pages, 11 point font size, including figures, tables and references. The total number of figures and tables may be up to 10 (sub-figures in (a), (b), (c), etc. counted separately). The total number of references may be up to 30. There should be at least 1.25" margin on left and right sides, and 1" margin from top and bottom. The figures and tables should be placed in the center of the column, and not tightly embedded into the text column. For some special issues, the page limit may be even less if more papers are included. In any event, check with the guest editors before writing the full manuscript.

% White paper due Feb 20, up to 4 pages length
% more info: https://signalprocessingsociety.org/sites/default/files/uploads/special_issues_deadlines/SPM_SI_geo_sp.pdf

\title{****title*****}
\author[1,2]{Eileen R. Martin}
\author[2]{Fantine Huot}
\author[1]{Yinbin Ma}
\author[2]{Robert Cieplicki}
\author[1,2]{Biondo L. Biondi}
\affil[1]{Institute for Computational and Mathematical Engineering, Stanford University}
\affil[2]{Department of Geophysics, Stanford University}
%\date{}                                           % Activate to display a given date or no date

\begin{document}
\maketitle

\section*{Motivation}  *****2/3 page Eileen****

\section*{Significance} ******2/3 page Eileen****

\section*{Outline of the paper}
	\begin{enumerate}
	\item Abstract/summary
	\item Introduction 
		\begin{itemize}
		\item *****words about intro, motivation****
		\end{itemize}
	\item Related work
		\begin{enumerate}
		\item Background on ambient noise: how it works that for the cost of just laying down receivers, you can extract signals mimicking those of an active seismic survey with a source at any receiver ****add citations******
		\item Background on Distributed Acoustic Sensing: how it works ****add citations****, how it has been used so far ****add citations****
		\item Examples of ambient seismic noise studies using Distributed Acoustic Sensing *****add citations****
		\end{enumerate}
	\item Data Processing: we process the noise we record so that it mimicks data that would be recorded if the noise field followed theoretical ideals. Some common strategies include ****citations to review like papers*****
		\begin{enumerate}
		\item Clustering of data to speed up noise field exploration
		\item How to use earthquakes to estimate sensor coupling 
		\item Detection of localized, persistent noise sources 
		\item Removing vehicle noise: design of filter for cars, for each piece of noise find nearest cluster to see if it's a car, then filter out the car while preserving the use of other noise in the same time and place
		\end{enumerate}
	\item Results
		\begin{enumerate}
		\item Comparison of convergence rates of virtual source response function estimates ****cite Seats 2012**** (both cross-correlation and cross-coherence) for traditional processing and new processing methods
		\item Velocity model resulting from ambient noise tomography
		\end{enumerate}
	\item Conclusions
	\item Acknowledgments
	\end{enumerate}


%\begin{abstract} 
%*****abstract words here - 150-200 Eileen*****
%\end{abstract}


%\section*{Motivation}
%****what's ambient noise used for? - 200-250 Eileen****

%****why use DAS? brief overview of its use****



%\section*{New Contributions}
%****what's new that we're doing, how it's different from what's been done before - 150 Eileen, little Fantine \& Yinbin****



%\section*{Background Theory}
%****ambient noise background - 150-220 Eileen****



%\section*{Experimental Overview}
%****overview of the array, potential noise sources, ambient noise field - 200 Eileen****
%****Figure : 2 panels: Installation + map****



%\section*{Data Processing}
%***overview of data processing workflow - 200-250 Eileen****
%***Figure : stft plots at a few different spots***
%***Mention whitening here***
%***Mention earthqukes and quarry blasts here***

%\subsection*{Estimating Sensor Coupling Response}
%***talk about different responses from different sensors and how to compensate for these, use many earthquakes for relative responses****
%***100 Eileen***
%***Figure : Boxed earthquake recording + response versus channel***

%% \subsection*{Localizing Persistent Noise Sources}
%% ***talk about localizing these things (motors, traffic, etc....) and show a little beamforming in small 1D parts of the array bandpassed to these persistent frequency peaks***

%% \subsection*{Removing Earthquakes and Quarry Blasts}
%% ****talk about getting rid of big events lighting up whole array****

%\subsection*{Removing Vehicle Noise}
%****talk about clustering and comparing noise sources****
%***500 Fantine \& Yinbin***
%***Figures : Before/after car mute + clusters and projection on map***
%***Little Eileen for sentence about why cars are a problem***

%% \subsection*{Removing Correlated Noise Sources}
%% *****talk about detecting temporally and spatially correlated noise sources, more generally than cars****

%% \subsection*{Traveltime Picking}
%% *** talk about traveltime picking before Rayleigh wave tomography****
%% *** STANDBY *** 

%% \subsection*{Tomography}
%% **** talk about how near-surface tomography is set up****
%% *** STANDBY *** 

%\section*{Results}
%***overview of results****
%*** Convergence rates
%***150 Eileen *** 
%***Figure : line plot of correlation coeeficients vs epoch/time***

%% \subsection*{Convergence Rate Improvement Due to Pre-Processing}
%***show comparisons of cross-correlations with/without sensor coupling response, and with scaling by energy per channel of cross-correlations (also show spectrum per channel of these)****
%\\
%******show comparisons of convergence rates with/without vehicles removed, and with vehicles just removed via STA/LTA jumps****


%% \subsection*{Tomography Along Array Lines}
%% ***show tomography results along multiple lines on campus****
%% *** STANDBY *** 

%\section*{Summary}
%****summary words****
%***100 Eileen ***

\section*{Biographies}
\textbf{Eileen R. Martin} \texttt{ermartin@stanford.edu} is a Ph.D. student in Stanford's Institute for Computational and Mathematical Engineering, and an M.S. student in the geophysics department. She first started working with ambient noise and DAS data from permafrost thaw monitoring experiments done by Lawrence Berkeley National Lab and the US Army Corps of Engineers.
\\
\textbf{Fantine Huot} \texttt{**email address} *****
\\
\textbf{Yinbin Ma} \texttt{**email address} *****
\\
\textbf{Robert Cieplicki} \texttt{***email address} ****
\\
\textbf{Steve Cole} \texttt{Steve.Cole@optasense.com} is Manager of Integrated Analysis at OptaSense. He has worked in the oil and gas industry for over 30 years at companies including Fugro, Chevron, and 4th Wave Imaging, which he co-founded. He has a Ph.D. in geophysics from Stanford.
\\
\textbf{Biondo L. Biondi} \texttt{**email address} ****

\begin{thebibliography}{30} % no more than 30 references

% example of trenched ambient noise recording with distributed acoustic sensing
\bibitem{AjoFranklin2015} J. Ajo-Franklin, N. Lindsey, S. Dou, T.M. Daley, B. Freifeld, E.R. Martin, M. Robertson, C. Ulrich and A. Wagner, "A Field Test of Distributed Acoustic Sensing for Ambient Noise Recording," \textit{SEG Technical Program Expanded Abstracts}, pp. 2620-2624, 2015.

% working with DAS data, how DAS works basically
\bibitem{Bakku2015} S.K. Bakku, "Fracture Characterization from Seismic Measurements in a Borehole," Ph.D. Thesis, Massachusetts Institute of Technology, 2015.

% temporal and spectral and one bit normalization
\bibitem{Bensen2007} G.D. Bensen, M.H. Ritzwoller, M.P. Barmin, A.L. Levshin, F. Lin, M.P. Moschetti, N.M. Shapiro and Y. Yang, "Processing seismic ambient noise data to obtain reliable broad-band surface wave dispersion measurements," \textit{Geophys. J. Int.}, vol. 169, no. ****, pp. 1239-1260, 2007.

% whitenging for robustness to velocity changes
\bibitem{Daskalakis2016} E. Daskalakis, C.P. Evangelidis, J. Garnier, N.S. Melis, G. Papanicolaou and C. Tsoga, "Robust seismic velocity change estimation using ambient noise recordings," \textit{Geophys. J. Int.}, vol. 205, no. 3, pp. 1926-1936, 2016.

% ambient noise with mine blasts (similar to earthquakes)
\bibitem{Girard2016} A.J. Girard and J. Shragge, "Extracting body waves from ambient seismic recordings," \textit{SEG Technical Program Expanded Abstracts}, pp. 2715-2719, 2016.

% sensitivity of DAS cables with slippage, straight or helical
\bibitem{Kuvshinov2015} B.N. Kuvshinov, "Interaction of helically wound fibre-optic cables with plane seismic waves," \textit{Geophysical Prospecting}, vol. ***, no. ***, pp. 1-18, 2015.

% using horizontal components in interferometry as Rayleigh and Love wave components
\bibitem{Lin2008} F. Lin, M.P. Moschetti and M.H. Ritzwoller, "Surface wave tomography of the western United States from ambient seismic noise: Rayleigh and Love wave phase velocity maps," \textit{Geophys. J. Int.} vol. 173, no. ***, pp. 281-298, 2008.

% example of interferometry of ambient noise recorded by a trenched DAS array
\bibitem{Martin2015} E.R. Martin, J. Ajo-Franklin, S. Dou, N. Lindsey, T.M. Daley, B. Freifeld, M. Robertson, A. Wagner, C. Ulrich, "Interferometry of ambient noise from a trenched distributed acoustic sensing array," \textit{SEG Technical Program Expanded Abstracts}, pp. 2445-2450, 2015.

% example of roadside DAS interferometry
\bibitem{Martin2016} E.R. Martin, N.J. Lindsey, S. Dou, J. Ajo-Franklin, T.M. Daley, B. Freifeld, M. Robertson, C. Ulrich, A. Wagner and K. Bjella, "Interferometry of a roadside DAS array in Fairbanks, AK," \textit{SEG Technical Program Expanded Abstracts}, pp. 2725-2629, 2016.


% cross correlation, coherence and deconvolution, and using traffic noise
\bibitem{Nakata2011} N. Nakata, R. Snieder, T. Tsuji, K. Larner and T. Matsuoka, "Shear wave imaging from traffic noise using seismic interferometry by cross-coherence," \textit{Geophysics}, vol. 76, no. 6, pp. SA97-SA106, 2011.

% ambient noise on dense array in Long Beach
\bibitem{Nakata2015} N. Nakata, J.P. Chang, J.F. Lawrence and P. Bou\'{e}, "Body wave extraction and tomography at Long Beach, California, with ambient-noise interferometry," \textit{J. Geophys. Res. Solid Earth}, vol. 120, pp. 1159-1173, 2015.

% how DAS works
\bibitem{Posey2000} R. Posey Jr., G.A. Johnson and S.T. Vohra, "Strain sensing based on coherence Rayleigh scattering in an optical fibre," \textit{Electronics Letters}, vol. 36, no. 20, pp. 1688-1689, 2000.

% overlapping windows, method of measuring convergence with correlation coefficient
\bibitem{Seats2012} K.J. Seats, J.F. Lawrence and G.A. Prieto, "Improved ambient noise correlation functions using Welch's method," \textit{Geophys. J. Int.}, vol. 188, no. ***, pp. 513-523, 2012.

% fundamental paper in ambient noise
\bibitem{Lobkis2001} O.I. Lobkis and R.L. Weaver, "On the emergence of the Green's function in the correlations of a diffuse field," \textit{J. Acoust. Soc. Am.}, vol. 110, no. 6, pp. 3011-3017, 2001.

% DAS conversion to velocities, and VSP example and side-by-side multi-mode, single-mode and 3C geophones
\bibitem{Miller2016} D.E. Miller, T.M. Daley, B.M. Freifeld, M. Robertson, J. Cocker and M. Craven, "Simultaneous Acquisition of Distributed Acoustic Sensing VSP with Multi-mode and Single-mode FIber-optic Cables and 3C-Geophones at the Aquistore CO$_2$ Storage Site," \textit{CSEG Recorder}, pp. 28-***, June 2016.

% basics of interferometry
\bibitem{Wapenaar2010A} K. Wapenaar, D. Draganov, R. Snieder, X. Campman and A. Verdel, "Tutorial on seismic interferometry: Part 1 - Basic principles and applications," \textit{Geophysics}, vol. 75, no. 5, pp. 75A195-75A209, 2010.
% more basics of interferometry
\bibitem{Wapenaar2010B} K. Wapenaar, E. Slob, R. Snieder and A. Curtis, "Tutorial on seismic interferometry: Part 2 - Underlying theory and new advances," \textit{Geophysics}, vol. 75, no. 5, pp. 75A211-75A227, 2010.

% example of DAS for microseismic event detection
\bibitem{Webster2013} P. Webster, J. Wall, C. Perkins and M. Molenaar, "Micro-Seismic Detection using Distributed Acoustic Sensing," \textit{SEG Technical Program Expanded Abstracts}, pp. 2459-2463, 2013.

% another example of DAS interferometry, this one side-by-side with geophones
\bibitem{Zeng2017} X. Zeng, C. Thurber, H. Wang, D. Fratta, E. Matzel, PoroTomo Team, "High-resolution Shallow Structure Revealed with Ambient Noise Tomography on a Dense Array," \textit{Proceedings, 42nd Workshop on Geothermal Reservoir Engineering}, Stanford University, Stanford, CA, Feb. 13-15, 2017, SGP-TR-212.

% velocity artifacts from ambient noise source changes
\bibitem{Zhan2013} Z. Zhan, V.C. Tsai and R.W. Clayton, "Spurious velocity changes caused by temporal variations in ambient noise frequency content," \textit{Geophys. J. Int}, vol. 194, no. ***, pp. 1574-1581, 2013.

\end{thebibliography}

\end{document}  
