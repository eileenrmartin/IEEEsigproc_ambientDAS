\documentclass[11pt]{article}
\usepackage[margin=1in]{geometry}                % See geometry.pdf to learn the layout options. There are lots.
\geometry{letterpaper}                   % ... or a4paper or a5paper or ... 
%\geometry{landscape}                % Activate for for rotated page geometry
%\usepackage[parfill]{parskip}    % Activate to begin paragraphs with an empty line rather than an indent
\usepackage{graphicx}
\usepackage{amssymb}
\usepackage{epstopdf}
\DeclareGraphicsRule{.tif}{png}{.png}{`convert #1 `dirname #1`/`basename #1 .tif`.png}
\usepackage{authblk}
\usepackage{titlesec}

%For final version of Special issue article: For each paper in the special issue, up to 20 single column double-spaced pages, 11 point font size, including figures, tables and references. The total number of figures and tables may be up to 10 (sub-figures in (a), (b), (c), etc. counted separately). The total number of references may be up to 30. There should be at least 1.25" margin on left and right sides, and 1" margin from top and bottom. The figures and tables should be placed in the center of the column, and not tightly embedded into the text column. For some special issues, the page limit may be even less if more papers are included. In any event, check with the guest editors before writing the full manuscript.

% White paper due Feb 20, up to 4 pages length
% more info: https://signalprocessingsociety.org/sites/default/files/uploads/special_issues_deadlines/SPM_SI_geo_sp.pdf

% KEYWORDS from IEEE Taxonomy 2017:
% Sensors: Acoustic Sensors
% Sensors: Optical fiber sensors
% Signal processing: Acoustic signal processing
% Signal processing: Array signal processing
% Signal processing: Geophysical signal processing
% Signal processing: Multidimensional signal processing
% Science-general: Waves, Seismic waves
% Science-general: Geophysics, Surface waves

\title{\vspace{-1.8cm} \Large Demystifying Ambient Seismic Noise Analysis: \\Automating processing decisions in the presence of coherent noise \vspace{-0.5cm}}
\author[1,2]{\small Eileen R. Martin}
\author[2]{Fantine Huot}
\author[1]{Yinbin Ma}
\author[2]{Robert Cieplicki}
\author[3]{Steve Cole}
\author[3]{Martin Karrenbach}
\author[2,1]{Biondo L. Biondi \vspace{-0.4cm}}
\affil[1]{Institute for Computational and Mathematical Engineering, Stanford University, Stanford, CA}
\affil[2]{Department of Geophysics, Stanford University, Stanford, CA}
\affil[3]{OptaSense, Brea, CA}
\date{}                                

\begin{document}
\maketitle

\vspace{-1.8cm}
\subsection*{Motivation} 
\vspace{-0.2cm}
By measuring the speed of seismic waves propagating in the Earth's near-surface, we can image the top tens to hundreds of meters of the subsurface, with deeper features being resolved by lower frequencies. These seismic velocity images can be interpreted to evaluate earthquake or landslide risk, to detect permafrost, to find sinkholes or tunnels, or to track near surface changes related to drilling activities. Additionally, in cases of complex near-surface conditions, resolving this complexity is a prerequisite to obtaining a high-quality image of the deeper subsurface.
\par
By cross-correlating noise recorded at a selected receiver with noise recorded by all other receivers in an array, we can extract signals mimicking an active seismic survey with a source at the selected receiver, called its virtual source response function \cite{Lobkis2001}, \cite{Lin2008}, \cite{Wapenaar2010A}, \cite{Wapenaar2010B}. Thus, when active sources are too costly or logistically prohibitive, passive seismic can be a good option for near-surface imaging. However, the theory is limited by the assumption of homogeneous uncorrelated sources \cite{Wapenaar2010A}. Non-ideal sources can cause artifacts in extracted velocities, but with careful processing experts can overcome these limitations \cite{Bensen2007}, \cite{Daskalakis2016}, \cite{Zhan2013}, even for anthropogenic sources occurring at higher frequencies \cite{Girard2016}, \cite{Martin2015}, \cite{Martin2016}, \cite{Nakata2011}, \cite{Nakata2015}, \cite{Zeng2017}. Acquisition is a further issue: nearly all of the past ambient noise studies on dense arrays have been temporary arrays which were labor-intensive to install and maintain \cite{Lin2008}, \cite{Nakata2011}, \cite{Nakata2015}, \cite{Zeng2017}. 
\par
Distributed acoustic sensing (DAS) is a new acquisition technology being increasingly adopted in the energy industry for microseismic monitoring \cite{Webster2013} and time-lapse seismic surveys \cite{Daley2013}, \cite{Mateeva2014}, \cite{Bakku2015}, \cite{Miller2016}. DAS probes a fiber optic cable with a laser interrogator unit (IU) to repurpose that fiber as a series of strain sensors. Most experiments use low-cost straight fibers, but these fibers only measure strain along the axis of the fiber \cite{Kuvshinov2016}. This downside is often outweighed by the benefit of a dense series of permanently installed seismic sensors communicating over long distances and running on a single power source.
\par
Motivated by avoiding the maintenance cost of node arrays, there have been several recent ambient noise experiments using fiber optics buried in shallow trenches directly coupled to the ground \cite{AjoFranklin2015}, \cite{Martin2015}, \cite{Martin2016}, \cite{Zeng2017}. In fact, we can sacrifice some ground-to-sensor coupling in favor of easier installation by running fibers in existing telecommunications conduits \cite{Martin2017}. By running fibers in existing conduits, or even plugging into unused fibers in existing telecommunications bundles, easy, on-demand, repeatable seismic studies (even in urban areas) will soon be a reality. When the IU probes a fiber, the data can serve many purposes: near-surface imaging with ambient noise, earthquake detection \cite{Martin2017} and location, studying local site response to earthquakes, tracking vehicle traffic, potentially even detecting underground tunneling, studying the status of fluid pump systems nearby, or detecting intrusion into manholes. 
\par
With ambient noise data becoming increasingly easy to record, data volumes are increasing, and we can only extract their full value if we further automate the processing workflow.

\vspace{-0.5cm}
\subsection*{Significance} 
\vspace{-0.2cm}
This manuscript will walk through the basics of processing ambient noise for near-surface imaging, and introduce some tools that are new to this application area, but already used for other spatial and temporal data. Its purposes: (1) overview of ambient noise processing for statisticians or mathematicians to encourage them to develop data analysis tools or theory for working with non-ideal ambient noise, (2) show optical engineers an application where an optical measurement device's noise level and sensitivity should be improved, and (3) expose geophysicists to an emerging acquisition method and a few machine learning tools and metrics to quantify processing decisions, a task that will become increasingly more difficult as data acquisition becomes easier and cheaper.
\par 
It is difficult to build intuition about ambient noise processing without looking at real data, so we will present this tutorial in the context of a case study: a figure-eight-shaped array of 2.4 km of fiber optics lying loosely in existing telecommunications conduits underneath the Stanford University campus. This particular experiment is ideal because it demonstrates the wide variety of issues coming up on the horizon as we push for broader use of ambient noise. 
\par
We will discuss the noise and sensitivities particular to recording with DAS, and a novel deployment that can greatly reduce costs: fiber loosely laying in existing telecommunications conduits. The array detects a wide variety of seismic noise sources that do not conform to the ideals of existing ambient noise theory: it sits in a seismically active region, 20 km from the Pacific ocean, 7 km from the San Francisco bay, with major highways on either side, a variety of roads with differing levels of traffic near the fiber, regular quarry blasts within 15 km, plumbing and HVAC systems throughout the site, multiple construction sites near the array, and foot and bicycle traffic throughout. With over 600 sensors continuously recording 50 samples per second since September 2016, manual inspection of most data is infeasible, making the automation tools shown in this paper critical to extracting subsurface information from the data.

\vspace{-0.5cm}
\subsection*{Outline of the paper}
	\begin{enumerate}
	\vspace{-0.2cm}
	\item Abstract/summary
	\vspace{-0.3cm}
	\item Introduction: (i) use of near surface imaging, (ii) tradeoffs: sensor quality vs. cost and array density (DAS or MEMS \cite{Evans2014}, etc...), (iii) idea of ambient noise to get rid of active sources, (iv) dense data acquisition over longer periods of time (potentially continuously streaming in as it's recorded) is becoming easy, so we must further automate ambient noise processing.
	\vspace{-0.3cm}
	\item Related work:
		\begin{enumerate}
		\vspace{-0.3cm}
		\item Background on ambient noise: how you can extract signals mimicking those of an active seismic survey with a source at any receiver \cite{Lobkis2001}, \cite{Lin2008}, \cite{Wapenaar2010A}, \cite{Wapenaar2010B}, how non-ideal noise sources can cause artifacts in extracted velocities, and with careful processing choices this can be done in the presence of non-ideal noise sources \cite{Bensen2007}, \cite{Daskalakis2016}, \cite{Zhan2013}, even higher frequencies sources \cite{Girard2016}, \cite{Martin2015}, \cite{Martin2016}, \cite{Nakata2011}, \cite{Nakata2015}, \cite{Zeng2017}
		\vspace{-0.1cm}
		\item Background on Distributed Acoustic Sensing: how it works \cite{Bakku2015}, \cite{Posey2000}, how it has been used so far \cite{Bakku2015}, \cite{Daley2013}, \cite{Webster2013}, \cite{Mateeva2014}, and its limitations \cite{Kuvshinov2016}
		\vspace{-0.1cm}
		\item Examples of ambient noise studies with DAS \cite{AjoFranklin2015}, \cite{Martin2015}, \cite{Martin2016}, \cite{Martin2017}, \cite{Zeng2017}
		\vspace{-0.1cm}
		\item Examples of how machine learning techniques have improved accuracy and sped up processing in other seismic applications \cite{Fisher2016}, \cite{Yoon2015}
		\end{enumerate}
	\vspace{-0.4cm}
	\item Data Processing on recorded noise so it mimics data from theoretically ideal noise field \cite{Bensen2007}
		\begin{enumerate}
		\vspace{-0.3cm}
		\item New: tools to aid people in speeding up noise field exploration and identification of potential issues including clustering/separation of data with features including wavelet transforms (k-means, hierarchical agglomerative, SVM separation) \cite{Liao2005}, \cite{Hastie2009}, \cite{Mallat2008} 
		\vspace{-0.1cm}
		\item Common tool: STA/LTA \cite{Withers1998} in time and Fourier domain to find earthquakes or quarry blasts \cite{Bensen2007}, \cite{Girard2016}, may better detect small events efficiently with fingerprint search \cite{Yoon2015}
		\vspace{-0.1cm}
		\item New: How to use earthquakes to estimate sensor coupling heterogeneity when array installation is relatively uncontrolled (loose fiber laying in variety of conduits) 
		\vspace{-0.1cm}
		\item New: Automatically detect \& filter persistent noise sources from short time spectra
		\vspace{-0.1cm}
		\item New: Automate the design of f-k domain filters for particular user-selected clusters based on separation from other clusters. In particular, a demonstration of removing vehicle noise: time domain mute vs. manually designed f-k vs. automated design. 
		\vspace{-0.1cm}
		\item New: In presence of strong/correlated sources, cross-coherence tends to work better than cross-correlation in the sense of producing virtual source response estimates with fewer artifacts \cite{Nakata2011}, \cite{Martin2016}. Show metrics to automate choice of stack weighting of cross-correlation or cross-coherence per receiver pair (can change over time).
		\end{enumerate}
	\vspace{-0.5cm}
	\item Measuring improvements: to compare old and new processing methods, measure convergence of virtual source response function estimates (both cross-coherence and cross-correlation) as more data is incorporated using the correlation coefficient like \cite{Seats2012}, but use a random sampling scheme over all data so convergence rates have error bars.
	\vspace{-0.3cm}
	\item Show a few dispersion images and a velocity model resulting from surface wave tomography, a couple of the possible uses of the virtual source response estimates.
	\vspace{-0.3cm}
	\item Conclusions and future directions
	\end{enumerate}

\vspace{-0.9cm}
\subsection*{Biographies}
\small
\vspace{-0.2cm}
\textbf{Eileen R. Martin} (\texttt{ermartin@stanford.edu}) is a Ph.D. student in Stanford's Institute for Computational and Mathematical Engineering, and an M.S. student in the Geophysics Department. She is also affiliated with Lawrence Berkeley National Lab where she collaborates with a team on fiber optic monitoring of permafrost thaw and on the use of dark fiber for event detection.
\\
\textbf{Fantine Huot} (\texttt{fantine@sep.stanford.edu}) received her M.S. in Science and Engineering from the Ecole Normale Sup\'erieure des Mines de Paris in 2013. Following graduation, she worked as a software developer in France. In fall 2015, she started her Ph.D. in Geophysics at Stanford University and joined the Stanford Exploration Project. She leverages machine learning algorithms to address earth science problems.
\\
\textbf{Yinbin Ma} (\texttt{yinbin@stanford.edu}) is a Ph.D. student in the Institute for Computational and Mathematical Engineering at Stanford. He works on time-lapse seismic imaging at the Stanford Exploration Project.
\\
\textbf{Robert Cieplicki} (\texttt{robcie@stanford.edu}) is a PhD student in the Zoback Stress and Crustal Mechanics Research Group (Geophysics Department, Stanford University). He applies machine learning to horizontal wells drilled in shale hydrocarbon reservoirs to predict the efficiency of hydraulic fracturing.
\\
\textbf{Steve Cole} (\texttt{Steve.Cole@optasense.com}) is Manager of Integrated Analysis at OptaSense. He has worked in the oil and gas industry for over 30 years at companies including Fugro, Chevron, and 4th Wave Imaging, which he co-founded. He has a Ph.D. in geophysics from Stanford.
\\
\textbf{Martin Karrenbach} (\texttt{Martin.Karrenbach@optasense.com}) Martin Karrenbach is the senior manager of innovation at OptaSense in Brea, CA. He received his Ph.D. in geophysics from Stanford University in 1995. 
\\
\textbf{Biondo L. Biondi} (\texttt{biondo@stanford.edu}) Biondo Biondi is a professor in the geophysics department at Stanford University, and leads the Stanford Exploration Project. He received his Ph.D. in geophysics from Stanford in 1990. He published the first book on 3D reflection seismology ("3D Seismic Imaging") and pioneered many methods in wavefield migration and velocity analysis.

\vspace{-0.6cm}
\begingroup
\titleformat*{\section}{\large \bf}
\begin{thebibliography}{30} % no more than 30 references
\vspace{-0.3cm}
\small
% example of trenched ambient noise recording with distributed acoustic sensing
\bibitem{AjoFranklin2015} J. Ajo-Franklin, N. Lindsey, S. Dou, T.M. Daley, B. Freifeld, E.R. Martin, M. Robertson, C. Ulrich and A. Wagner, ``A Field Test of Distributed Acoustic Sensing for Ambient Noise Recording," \textit{SEG Technical Program Expanded Abstracts}, pp. 2620-2624, 2015.
\vspace{-0.2cm}
% working with DAS data, how DAS works basically
\bibitem{Bakku2015} S.K. Bakku, ``Fracture Characterization from Seismic Measurements in a Borehole," Ph.D. Thesis, Massachusetts Institute of Technology, 2015.
\vspace{-0.2cm}
% temporal and spectral and one bit normalization
\bibitem{Bensen2007} G.D. Bensen, M.H. Ritzwoller, M.P. Barmin, A.L. Levshin, F. Lin, M.P. Moschetti, N.M. Shapiro and Y. Yang, ``Processing seismic ambient noise data to obtain reliable broad-band surface wave dispersion measurements," \textit{Geophys. J. Int.}, vol. 169, no. 3, pp. 1239-1260, 2007.
\vspace{-0.2cm}
% whitenging for robustness to velocity changes
\bibitem{Daskalakis2016} E. Daskalakis, C.P. Evangelidis, J. Garnier, N.S. Melis, G. Papanicolaou and C. Tsoga, ``Robust seismic velocity change estimation using ambient noise recordings," \textit{Geophys. J. Int.}, vol. 205, no. 3, pp. 1926-1936, 2016.
\vspace{-0.2cm}
% nice intro to DAS with a seismic active experiment
\bibitem{Daley2013} T.M. Daley, B.M. Freifeld, J. Ajo-Franklin and S. Dou, ``Field testing of fiber-optic distributed acoustic sensing (DAS) for subsurface seismic monitoring," \textit{The Leading Edge}, pp. 936-942, June 2013.
\vspace{-0.2cm}
% an overview comparison of MEMS accelerometers
\bibitem{Evans2014} J.R. Evans, R.M. Allen, A.I. Chung, E.S. Cochran, R. Guy, M. Hellweg and J.F. Lawrence, ``Performance of Several Low-Cost Accelerometers," \textit{Seismol. Res. Lett.}, vol. 85, no. 1, pp. 148-158, 2014.
\vspace{-0.2cm}
% another example of passive seismic data being analyzed with machine learning techniques (including CWT as important feature)
\bibitem{Fisher2016} W.D. Fisher, T.K. Camp and V.V. Krzhizhanovskaya, ``Anomaly detection in earth dam and levee passive seismic data using support vector machines and automatic feature selection," \textit{J. Comput. Sci.}, in press, doi: 10.1016/j.jocs.2016.11.016, 2016.
\vspace{-0.2cm}
% ambient noise with mine blasts (similar to earthquakes)
\bibitem{Girard2016} A.J. Girard and J. Shragge, ``Extracting body waves from ambient seismic recordings," \textit{SEG Technical Program Expanded Abstracts}, pp. 2715-2719, 2016.
\vspace{-0.2cm}
% lots of basic data analysis tools
\bibitem{Hastie2009} T. Hastie, R. Tibshirani and J. Friedman, \textit{The Elements of Statistical Learning: Data Mining, Inference, and Prediction}, 2nd ed. New York, NY, USA: Springer, 2009.
\vspace{-0.2cm}
% sensitivity of DAS cables with slippage, straight or helical
\bibitem{Kuvshinov2016} B.N. Kuvshinov, ``Interaction of helically wound fibre-optic cables with plane seismic waves," \textit{Geophysical Prospecting}, vol. 64, no. 3, pp. 671-688, 2016.
\vspace{-0.2cm}
% review on clustering of time series data
\bibitem{Liao2005} T.W. Liao, ``Clustering of time series data," \textit{Pattern Recogn.}, vol. 38, no. 11, pp. 1857-1874, 2005.
\vspace{-0.2cm}
% using horizontal components in interferometry as Rayleigh and Love wave components
\bibitem{Lin2008} F. Lin, M.P. Moschetti and M.H. Ritzwoller, ``Surface wave tomography of the western United States from ambient seismic noise: Rayleigh and Love wave phase velocity maps," \textit{Geophys. J. Int.}, vol. 173, no. 1, pp. 281-298, 2008.
\vspace{-0.2cm}
% fundamental paper in ambient noise
\bibitem{Lobkis2001} O.I. Lobkis and R.L. Weaver, ``On the emergence of the Green's function in the correlations of a diffuse field," \textit{J. Acoust. Soc. Am.}, vol. 110, no. 6, pp. 3011-3017, 2001.
\vspace{-0.2cm}
% wavelet transforms and basic signal processing reference
\bibitem{Mallat2008} S. Mallat, \textit{A Wavelet Tour of Signal Processing}, 3rd ed. Burlington, MA, USA: Academic Press, 2008.
\vspace{-0.2cm}
% example of interferometry of ambient noise recorded by a trenched DAS array
\bibitem{Martin2015} E.R. Martin, J. Ajo-Franklin, S. Dou, N. Lindsey, T.M. Daley, B. Freifeld, M. Robertson, A. Wagner, C. Ulrich, ``Interferometry of ambient noise from a trenched distributed acoustic sensing array," \textit{SEG Technical Program Expanded Abstracts}, pp. 2445-2450, 2015.
\vspace{-0.2cm}
% example of roadside DAS interferometry
\bibitem{Martin2016} E.R. Martin, N.J. Lindsey, S. Dou, J. Ajo-Franklin, T.M. Daley, B. Freifeld, M. Robertson, C. Ulrich, A. Wagner and K. Bjella, ``Interferometry of a roadside DAS array in Fairbanks, AK," \textit{SEG Technical Program Expanded Abstracts}, pp. 2725-2629, 2016.
\vspace{-0.2cm}
% overview of the array, reasons to use DAS in existing telecomm conduits
\bibitem{Martin2017} E.R. Martin, B.L. Biondi, M. Karrenbach, S. Cole, ``Continuous Subsurface Monitoring by Passice Seismic with Distributed Acoustic Sensors- The `Stanford Array' Experiment," \textit{Proceedings of the First EAGE Workshop on Practical Reservoir Monitoring}, 2017.
\vspace{-0.2cm}
\bibitem{Mateeva2014} A. Mateeva, J. Lopez, H. Potters, J. Mestayer, B. Cox, D. Kiyashchenko, P. Wills, S. Grandi, K. Hornman, B. Kuvshinov, W. Berlang, Z. Yang, R. Detomo, ``Distributed acoustic sensing for reservoir monitoring with vertical seismic profiling," \textit{Geophysical Prospecting}, vol. 62, no. 4, pp. 679-692, 2014.
\vspace{-0.2cm}
% DAS conversion to velocities, and VSP example and side-by-side multi-mode, single-mode and 3C geophones
\bibitem{Miller2016} D.E. Miller, T.M. Daley, B.M. Freifeld, M. Robertson, J. Cocker and M. Craven, ``Simultaneous Acquisition of Distributed Acoustic Sensing VSP with Multi-mode and Single-mode FIber-optic Cables and 3C-Geophones at the Aquistore CO$_2$ Storage Site," \textit{CSEG Recorder}, vol. 41, no. 6, pp. 28-33, 2016.
\vspace{-0.2cm}
% cross correlation, coherence and deconvolution, and using traffic noise
\bibitem{Nakata2011} N. Nakata, R. Snieder, T. Tsuji, K. Larner and T. Matsuoka, ``Shear wave imaging from traffic noise using seismic interferometry by cross-coherence," \textit{Geophysics}, vol. 76, no. 6, pp. SA97-SA106, 2011.
\vspace{-0.2cm}
% ambient noise on dense array in Long Beach
\bibitem{Nakata2015} N. Nakata, J.P. Chang, J.F. Lawrence and P. Bou\'{e}, ``Body wave extraction and tomography at Long Beach, California, with ambient-noise interferometry," \textit{J. Geophys. Res. Solid Earth}, vol. 120, pp. 1159-1173, 2015.
\vspace{-0.2cm}
% how DAS works
\bibitem{Posey2000} R. Posey Jr., G.A. Johnson and S.T. Vohra, ``Strain sensing based on coherence Rayleigh scattering in an optical fibre," \textit{Electronics Letters}, vol. 36, no. 20, pp. 1688-1689, 2000.
\vspace{-0.2cm}
% overlapping windows, method of measuring convergence with correlation coefficient
\bibitem{Seats2012} K.J. Seats, J.F. Lawrence and G.A. Prieto, ``Improved ambient noise correlation functions using Welch's method," \textit{Geophys. J. Int.}, vol. 188, no. 2, pp. 513-523, 2012.
\vspace{-0.2cm}
% basics of interferometry
\bibitem{Wapenaar2010A} K. Wapenaar, D. Draganov, R. Snieder, X. Campman and A. Verdel, ``Tutorial on seismic interferometry: Part 1- Basic principles and applications," \textit{Geophysics}, vol. 75, no. 5, pp. 75A195-75A209, 2010.
\vspace{-0.6cm}
% more basics of interferometry
\bibitem{Wapenaar2010B} K. Wapenaar, E. Slob, R. Snieder and A. Curtis, ``Tutorial on seismic interferometry: Part 2 - Underlying theory and new advances," \textit{Geophysics}, vol. 75, no. 5, pp. 75A211-75A227, 2010.
\vspace{-0.2cm}
% example of DAS for microseismic event detection
\bibitem{Webster2013} P. Webster, J. Wall, C. Perkins and M. Molenaar, ``Micro-Seismic Detection using Distributed Acoustic Sensing," \textit{SEG Technical Program Expanded Abstracts}, pp. 2459-2463, 2013.
\vspace{-0.2cm}
% STA/LTA for trigger algorithms
\bibitem{Withers1998} M. Withers, R. Aster, C. Young, J. Beiriger, M. Harris, S. Moore and J. Trujillo, ``A comparison of select trigger algorithms for automated global seismic phase and event detection," \textit{B. Seismol. Soc. Am.}, vol. 88, no. 1, pp. 95-106, 1998.
\vspace{-0.2cm}
% earthquake detection with templat ematching and machine learning
\bibitem{Yoon2015} C.E. Yoon, O. O'Reilly, K.J. Bergen and G.C. Beroza, ``Earthquake detection through computationally efficient similarity search," \textit{Science Advances}, vol. 1, no. 11, pp. e1501057, 2015.
\vspace{-0.2cm}
% another example of DAS interferometry, this one side-by-side with geophones
\bibitem{Zeng2017} X. Zeng, C. Thurber, H. Wang, D. Fratta, E. Matzel, PoroTomo Team, ``High-resolution Shallow Structure Revealed with Ambient Noise Tomography on a Dense Array," \textit{Proceedings, 42nd Workshop on Geothermal Reservoir Engineering}, Stanford University, CA, Feb. 13-15, 2017, SGP-TR-212.
\vspace{-0.2cm}
% velocity artifacts from ambient noise source changes
\bibitem{Zhan2013} Z. Zhan, V.C. Tsai and R.W. Clayton, ``Spurious velocity changes caused by temporal variations in ambient noise frequency content," \textit{Geophys. J. Int}, vol. 194, no. 3, pp. 1574-1581, 2013.

\end{thebibliography}
\endgroup

\end{document}  
